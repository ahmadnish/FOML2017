\documentclass[12pt, a4paper]{scrartcl}
\usepackage[english]{babel}
\usepackage{natbib}
\usepackage{url}
\usepackage{lmodern}        % Latin Modern family of fonts
\usepackage[T1]{fontenc}    % fontenc is oriented to output, that is, what fonts to use for printing characters. 
\usepackage[utf8]{inputenc}
\usepackage{amsmath}
\usepackage{amssymb}
\usepackage{graphicx}
\graphicspath{{../figs/}}
\usepackage{subcaption}
%\usepackage{parskip}
\usepackage{fancyhdr}
\usepackage{vmargin}
\usepackage{hyperref}
\usepackage{cleveref}
\usepackage{csquotes}
\usepackage{bm}
\usepackage{placeins}
\usepackage{pdfpages}
\usepackage{blindtext}
\usepackage[list-final-separator={, and }]{siunitx}
\usepackage{mathtools}

%\setmarginsrb{3 cm}{2.5 cm}{3 cm}{2.5 cm}{1 cm}{1.5 cm}{1 cm}{1.5 cm}
\setmarginsrb{3 cm}{1.5 cm}{3 cm}{1.5 cm}{1 cm}{0.5 cm}{1 cm}{0.5 cm}
\title{Breast Cancer Wisconsin Data Set} % Title
\author{Jannik Lukas Kossen \& Ahmad Neishabouri}                               % Author
\date{\today}                                         % Date

\makeatletter
\let\thetitle\@title
\let\theauthor\@author
\let\thedate\@date
\makeatother

\pagestyle{fancy}
\fancyhf{}
\rhead{\theauthor}
\lhead{Fundamentals of Machine Learning}
\cfoot{\thepage}

% Fix BibTex error
\setcitestyle{numbers}

% Define argmin, max
\DeclareMathOperator*{\argmin}{arg\,min}
\DeclareMathOperator*{\argmax}{arg\,max}
\DeclarePairedDelimiter\norm{\lVert}{\rVert}

\begin{document}

%%%%%%%%%%%%%%%%%%%%%%%%%%%%%%%%%%%%%%%%%%%%%%%%%%%%%%%%%%%%%%%%%%%%%%%%%%%%%%%%%%%%%%%%%

\begin{titlepage}
    \centering
%    \vspace*{0.5 cm}
    \includegraphics[scale = 0.6]{hdlogo}\\[2.0 cm]  % University Logo
     \begin{flushleft}
     \large  \hspace{1cm} A report on the 
	\end{flushleft}      
     \centering
    \rule{\linewidth}{0.2 mm} \\[0.4 cm]
    { \huge \bfseries \thetitle}\\
    \rule{\linewidth}{0.2 mm} \\[1.5 cm]
    
    \textsc{\LARGE Final Project}\\[0.5 cm]               % Course Code
    \textsc{\Large Heidelberg Collaboratory \\[0.5em] for Image Processing}\\[2.0 cm]  % University Name
    \thedate
   	\\[3em]
    \large
            \emph{Submitted to:}\\[1em]
            Prof.\ Dr.\ Ulrich Köthe \& Tutor: Lynton\\[1cm]
            \emph{Submitted by:} \\[1.5em]
            \begin{minipage}{0.4\textwidth}            
            	\begin{flushleft} 
					\large Jannik Lukas Kossen (JK) Student ID: 3495228 \\
					\small M.Sc. Physics, ungraded\\
      		  \end{flushleft}
      		 \end{minipage}
            \begin{minipage}{0.4\textwidth}            
            	\begin{flushright} 
    	    	    \large Ahmad Neishabouri (AN) Student ID: XXXXXX \\ 
         	   		\small M.Sc. Scientific Computing, graded
      		  \end{flushright}
      		 \end{minipage}\\[2 cm]


        
 
\end{titlepage}


\tableofcontents
\pagebreak

%TODO neish student id
%TODO fix titlepage (names and everything) 
%TODO centralise for PCA! dont normalise necessarily
%TODO reread algorithms in barbers book
%TODO figure out if imbalance is a problem
%TODO  put this in?
% Validation data is only needed when we compare different models. The cross validation might just yield a "lucky" winner. (Using different models on the test set can be seen as overfitting the test set. The winning model might have just been lucky. 


% How to include a figure.
%\begin{figure}
%	\centering
%	\includegraphics[width=0.7\textwidth]{hdlogo}
%	\caption{A figure.}
%	\label{fig:class-incremental}
%\end{figure}

\section{Introduction (JK)}
The following pages contain the report on our final project for the \emph{Fundamentals of Machine Learning} lecture given by Dr.\ Köthe in the last semester. As the lecture focused on the \emph{fundamentals} of machine learning we see this project as an opportunity to explore and revise again some fundamental techniques of machine learning and also dive into some of the algorithms which, while they were presented in the lectures, were not discussed (enough) in the tutorials.  Often, the application potential of these basic techniques on real-world tasks is rather small. However, they do constitute the foundations of more recent and better algorithms, which often embed or build upon these methods. 
The focus of this project shall therefore not be to \emph{solve} a task using fundamental machine learning, but rather to carefully study the different algorithms and compare their strengths and shortcomings in comparison with each other. To narrow the focus of the project a bit, we have chosen to only compare \emph{classification} methods.
We work with the \emph{Breast Cancer Wisconsin (Diagnostic) Data Set} \cite{street1993nuclear}, a binary classification task with the aim to identify whether a person does or does not have breast cancer based on a set of hand-crafted features. The data set should work very well for benchmarking algorithms, providing a nice balance between being well-behaved, i.\ e.\ solvable, and possessing enough depth and character to allow for some comparison between the algorithms, i.\ e.\ not being trivial to solve.


\paragraph{Data Set.} The abovementioned data set was first published alongside a paper called \emph{Nuclear Feature Extraction For Breast Cancer Diagnosis} \cite{street1993nuclear} in the 1993 edition of the \emph{International Symposium on Electronic Imaging: Science and Technology}, wherein a classification pipeline for breast cancer diagnosis is proposed. A small sample of skin tissue is extracted from the patients breast and analyzed using a camera mounted atop a microscope. Using an interactive tool the operator identifies individual cell nuclei, whose outlines are then semi-automatically determined. For each nucleus, ten features that are sought to describe the geometric shape and appearance of the nucleus are crafted. These features are the radius, perimeter, area, compactness, smoothness, concavity, concave points, symmetry, fractal dimension, and texture of the nucleus.
For the sake of brevity, the reader is referred to the original publication \cite{street1993nuclear} for a detailed description of how these features are calculated from the image and outline of a nucleus. 
The mean, max, and standard deviation values of these features is then calculated for the nuclei of a single sample, giving in total a 30-dimensional feature vector describing each cell nucleus.
The hope here is obviously, that these features (or a subset thereof) capture the differentiating aspect of tumor-ridden cell nuclei from their healthy counterparts. The data set is explored further in \cref{sec:expe}.

\paragraph{Structure.} The structure of this report is as follows. \Cref{sec:theo} discusses the theoretical foundations of each of the compared methods. In \cref{sec:expe}, unsupervised exploration techniques are employed to gain some understanding of the shape of our data set. The classification methods are applied and benchmarked. \Cref{sec:discu} then discusses these results in relation to peculiarities of each method and the data set.

As demanded by the faculty of Mathematics and Computer Science, each section or paragraph heading also contains a shorthand symbol, \emph{(JK)} or \emph{(AN)}, to indicate whether a certain section was written by Jannik Kossen or Ahmad Neishabouri.

\section{Theoretical Background}	
\label{sec:theo}

\paragraph{Preliminary Remarks (JK).} Note that, unless marked otherwise, the theoretical derivation and description attempt to follow the notation of the lecture \cite{kothe2018foml} for clarity.
This includes the convention to describe the data as $X = { X_1, X_2, \dots, X_N}$, where $N$ is the number of instances and each $X_i$ is a $D$-dimensional feature (row) vector. The feature $j$ of instance $i$ is therefore given by $X_{ij}$.  Index $i$ is solely used to index along the $N$-dimensional instance axis, whereas $j$ solely indexes the $D$-dimensional feature axis. The labels or ground truth instances of supervised learning are given as $Y_i$.

\paragraph{Principal Component Analysis (JK).} Principal Component Analysiss (PCA) is a simple yet effective dimension reduction technique. Often, it is also used exploratively to gain an understanding of the complexity of the data set. In PCA, we aim to map instances $X_i$ of the original feature space $\mathbb{R}^D$ to features $Z_i$ in a new feature space $\mathbb{R}^{D^\prime}$ via a linear mapping
\begin{align}
	Z_i = X_i A + b \, 
\end{align}
where $A$, a $D\times D^\prime$ matrix, and $b$ are the parameters of the linear mapping. We now define a mapping from $\mathbb{R}^{D^\prime}$  back to the original space $\mathbb{R}^D$ as
\begin{align}
	\tilde{X}_i = V Z_i + \mu \, ,
\end{align}
where for the sake of simplicity we assume $D^\prime=1$.
The ideal linear mapping between the two feature spaces is now found by demanding that the squared loss between the original $X_i$ and the reconstruction $\tilde{X}_i$ is minimized
\begin{align}
	\argmin_{V, \mu, \{Z_i\}} \: \sum_i \left( X_i - \tilde{X}_i \right)^2 \, .
\end{align}
By requiring centralized data $\bar{X}=0$, setting the arbitrary $\bar{Z}=0$, and requiring normalized $V^TV$, one finds the simplified version (see lecture for full derivation
\begin{align}
	&\argmin_{V, \{Z_i\}} \: \sum_i ( X_i - 
														\underbrace{V Z_i}_{\mathclap{=(X_i V^T)V}}
														)^2 \quad
														\text{s.t.} \enspace V^TV=1  \\
	&\argmax_{V, \lambda} \: VSV^T + \lambda \,  (1-VV^T) \, ,													\end{align}
where $\lambda$ is the Lagrange parameter for the constraint on $V$ and $S$ is the scatter matrix $S = \sum_i X_i^T X_I$. 
The derivation is presented in such detail here because an important insight can be gained  by deriving the above expression w.\ r\. t.\ $V$,
\begin{align}
	S V^T = \lambda V^T \, ,
\end{align}
$V^T$ is an eigenvector of S. Substituting this into the above problem simply yields
\begin{align}
	\argmax_\lambda \lambda \, .
\end{align}
The projection problem therefore has been reduced to finding the biggest eigenvector of S. In other words: if we want to linearly project our data into a one-dimensional feature space while retaining maximal information, we project along the direction of $V$, the vector belonging to the biggest eigenvalue $\lambda$ of the scatter matrix. Similarly, if we want to transform to a $D^\prime$ dimensional feature space, we project the original data instances along the $D^\prime$ largest eigenvectors $V_j$. Given by the optimization constraint above, the features (eigenvectors) of the new space are orthogonal. 

It can be shown that the projection vectors $V_j$, sorted by their eigenvalue size, project the data along those orthogonal axes that lie along the direction of maximal variance. This is because axes with a lot of variance contain a lot of information and therefore minimize the loss of information when transforming to the new feature space.


\paragraph{(k-) Nearest Neighbor Classification (JK).} One of the simplest approaches to classification is the the nearest neigbor (NN) classifier. Test set instances $i$ are simply classified with the same class as their \emph{nearest} neighbor from the training set (TS). 
Which neighbor is the \emph{nearest} is given by some distance metric $d(X_i, X_{i^\prime})$ between to instances $X_i$ and $X_{i^\prime}$. $d(X_i, X_{i^\prime})$ is often chosen to be simply the Euclidean distance.
In mathematical terms, the decision function for the NN classifier can be written as
\begin{align*}
	f_{\mathrm{NN}} = Y_i \, \text{,  where } \, i = \argmin_{i^\prime \in \mathrm{TS}} \: d(X_i, X_{i^\prime}) \, .
\end{align*}
The NN classifier requires the total memorization of the training set which becomes problematic for larger data sets.
A simple variant of the NN classifier is the k-nearest neighbor algorithm, which classifies instances by taking a majority vote of the classes of the surrounding k-nearest neighbors of the instance that is to be classified. In contrast to the simple NN classifier, the k-NN classifier is consistent, i.\ e.\ the k-NN classifier converges to the optimal Bayes classifier as $N\to\infty$.

\paragraph{Support Vector Machine (JK).} Out of all of the presented algorithms in the lecture, Support Vector Machines (SVMs) are among the most powerful and widely used. Similar to the previously presented linear discriminant analysis / logistic regression, in SVM decision planes are trained to discern between positive and negative instances of the data. When the data is linearly separable, i.\ e.\ there are multiple hyperplanes that perfectly separate the different classes, SVM chooses that hyperplane which maximizes the so-called \emph{safety margin}
\begin{align}
	&\max_{\beta, \beta_0} M \\
	= &	\max_{\beta, \beta_0} \left\{ \min_i Y_i ( X_i \beta + \beta_0) \right\} \, .
\end{align}
$X_i \beta + \beta_0$ gives the signed distance of instance $X_i$ to the decision plane parametrized by $(\beta, \beta_0)$. $Y_i (X_i \beta + \beta_0)$ is always positive. 
We therefore find the decision plane by maximizing the margin for the for the points closest to the decision plane. The obtained solution for an SVM depends entirely on those critical points, i.\ e.\ the decision plane of an SVM is governed by the points close to the decision plane, completely unlike the abovementioned LDA, where the majority of the points, which are close to the mean and not the boundary, decides the position of the decision plane. 
Setting the arbitrary $M=1$ and rescaling by the norm of $\beta$ yields the optimization problem of the linear SVM for separable data
\begin{align}
	\hat{\beta}, \hat{\beta_0} = \argmin_{\beta, \beta_0}  \norm{ \beta}^2 
	\quad \text{s.t.} \enspace \forall i: Y_i (X_i \beta + \beta_0 \ge 1) \, ,
\end{align}
where the constraint ensures that the decision line strictly separates data of the two classes.
Since real world data are usually not linearly separable, so-called \emph{slack} variables $s_i$ are introduced that represent a trade-off between the maximization of the margin and minimizing the loss over the training instances. 
By rearranging the constraint the linear SVM can be rewritten as
\begin{align}
	\hat{\beta}, \hat{\beta_0} = \argmin_{\beta, \beta_0} \frac{1}{2} \norm{ \beta }^2 + \frac{C}{N} \sum_i \text{loss} (X_i, Y_i | \beta, \beta_0) \, ,
\end{align}
where $C$ now represents the trade-off parameter between the two objectives. $C$ can be found using cross-validation. In SVM we also get to chose a loss term, where a popular choice is the \emph{hinge loss}
\begin{align}
	\text{Hinge Loss}(\hat{Y}_i, Y_i) =  \max(0, 1-\hat{Y}_i  Y_i) \, .
\end{align}
However, other types of losses are possible, with most popular choices producing similar outcomes, seeing as a lot of them behave similarly near the decision boundary.
From the above expression, a dual formulation can be obtained which is suited for a kernelized version of the SVM, which is not derived here.
Suffice to say, that to get to a kernel SVM, the above optimization problem has to be reformulated such that the data is only accessed via scalar products $X_i^T X_i^\prime$ which are then replaced by a kernel $K(X_i, X_i^\prime)$. This introduces a non-linearity that effectively maps the data to a higher dimensional space and allows one to solve more complex problems at the cost of introducing additional hyperparameters.
A popular kernel choice is the Gaussian kernel
\begin{align}
	K(X_i, X_i^\prime) = e^{-\gamma \norm{ X_i - X_i^\prime }^2} \, 
\end{align}
although other kernels such as polynomial or sigmoid kernels are also used.

The SVM optimization problem can be solved iteratively using various gradient descent methods.%

\paragraph{Multilayer Perceptron (JK).} Even though they were not part of the lecture, it seems wrong to not discuss neural networks at all. They have been hugely successful within the machine learning community, finding application in a broad variety of topics. While a lot of research is dedicated to \emph{convolutional neueral networks} in computer vision, simple neural networks can also be employed to do classification tasks.
This is why, even though cutting-edge neural networks are not discussed here, the multilayer perceptron (MLP), basic artificial neural network archticture, is introduced and applied. A \emph{very} brief description is given in the following paragraph, for a more detailed description see e.\ g.\ \cite{bishop2006prm}.
An MLP, as depicted in \cref{fig:mlp}, consists of several layers of neurons, where each layer consists of a column of neurons. Each neuron receives an input signal that is a weighted sum over all of the outputs of the neurons in the previous layer. A so-called bias is added to that sum and finally a non-linearity, such as a sigmoid function, is applied to give the output of a neuron. For a neuron $z_{1i} $ in the first hidden layer this would be
\begin{align*}
	o_{1i} = \text{sigmoid}\left( W_{1i} z_{0} + b_{1i} \right) \, ,
\end{align*}
where $W_{1i}$ are the weights connecting the inputs  $z_{0}$ to the neuron $i$ in the first hidden layer and $b_{1i}$ is the bias term. This operation is repeated all over the network. Any neural network therefore is a nested application of non-linear functions of the input data.

\begin{figure}
	\centering
	\includegraphics[width=0.7\textwidth]{mlp}
	\caption{Schematic diagram of a MLP. Image Credit \citep{saracoglu2010color}}
	\label{fig:mlp}
\end{figure}

For our binary classification problem, the output layer consists of two nodes, one for each of the classes. In training, the neural network is presented with examples from the training set and the output from the last layer is compared to the desired output. For binary classification problems, a binary cross entropy loss similar to LDA
\begin{align}
	\text{loss}(\{y_i, \hat{y_i}\}) = \frac{1}{N_b} \sum_{i=1}^{N_b} \left( y_i \log(\hat{y_i}) + (1-y_i)) \log(1-\hat{y_i}) \right)
\end{align}
is used. $N_b$ here is the batch size, which is not neccessarily equal to the number of instances $N$ in the data set. This loss can then be minimized in training, by means of \emph{backpropagation}: the partial derivatives of the loss with respect to the parameters are calculated and the parameters adjusted accordingly to minimize the loss.
Through many iterations of this, the network can be fit to the data set, although one needs to be careful not to overfit to the data.
Early stopping (stop the training process before convergence) and regularization techniques such as dropout (randomly drop a proportion of the neurons from the network)) may be used to combat overfitting.

\section{Experiments \& Discussion}
\label{sec:expe}
In the following, the tools and implementations from the python library scikit-learn \cite{scikit-learn} are heavily used. For the multilayer perceptron, the popular library tensorflow \cite{tensorflow2015-whitepaper} is employed.

\subsection{Exploratory Analysis \& Introduction to the Data Set (JK)}
Before the introduced classification methods are applied, it is crucial that some time is spent on exploring the \emph{Breast Cancer Wisconsin Data Set}.
The data set consists of $N=569$ instances with $D=30$ dimensions per instance. The data is somewhat imbalanced as \SI{63}{\percent} of the instances are false, i.\ e.\ come from benign cells without breast cancer. This might have to be accounted for in the analysis.
Data will be centered and standardized according to the training set distribution unless otherwise mentioned.
To get a feeling for the data set, 16 random combinations of two dimensions of the data set are shown in a scatter plot in \cref{fig:random_scatter}. While for some features, the different classes are clumped together, we can already see that for others a separation might be possible.
\begin{figure}
	\centering
	\includegraphics[width=\textwidth]{random_scatter}
	\caption{Shown are random 2D slices of the data set. Given in parenthesis are the dimensions that give the $(x,y)$ axis of the plot.}
	\label{fig:random_scatter}
\end{figure}

Following this hunch, a continued investigation of the data set is therefore pursued via principal component analysis.
The results of the principal component analysis are given in \cref{fig:pca}. We can see that the linear PCA is already fairly successful at separating the data set. Note, that as PCA is an unsupervised method, it has not been given any labels. Labels are only plotted for clarity. In \cref{fig:pca1} one can see that just by projecting the data among axes of high variance, the two class labels are somewhat separated. \Cref{fig:pca2} shows that the projection vectors actually point along axes of high variance in the original space. Note, that the two vectors being shown here only \emph{seem} to note be orthogonal, due to the fact that a 2D slice of the 30 dimensional feature space is shown. Finally, \Cref{fig:pca3} shows that the first component of the PCA can almost explain \SI{5}{\percent} of the variance in the data, if the first 6 components are combined this rises to \SI{90}{\percent}. 
The dimensions in \cref{fig:pca2} were chosen as the highest contributors to the first component of the PCA, which is why the also exhibit a high variance.

\begin{figure}
    \centering
    \begin{subfigure}[b]{0.3\textwidth}
        \includegraphics[width=\textwidth]{pca_1}
        \caption{}
        \label{fig:pca1}
    \end{subfigure}
    \begin{subfigure}[b]{0.3\textwidth}
        \includegraphics[width=\textwidth]{pca_3}
        \caption{}
        \label{fig:pca2}
    \end{subfigure}
    \begin{subfigure}[b]{0.3\textwidth}
        \includegraphics[width=\textwidth]{pca_2}
        \caption{}
        \label{fig:pca3}
    \end{subfigure}
    \caption{PCA of Data Set: (a) shows the two most important dimensions of the data set in the transformed space, (b) shows how the PCA projects along axes of maximal variance, the arrows indicate the direction of projection, (c) shows the explained variance of the original data set per PCA component.}\label{fig:pca}
\end{figure}

In conclusion, PCA can be successfully applied to this data set as a dimensionality reduction technique. The transformed, lower-dimensional data can then be used as an input to other classification algorithms. The success of the linear PCA also shows that the complexity of the data set is somewhat limited and that we can expect successful results from simple methods.

\subsection{Classification Algorithms}
\paragraph{Evaluation (JK).} In order to provide reliable performance analysis, 10-fold cross-validation as described in the lecture is employed for each algorithm. A strict separation of training and test set is ensured such that meaningful statements over the predictive power of the methods can be made.
Whenever an algorithm has hyperparameters an exhaustive grid search is performed in order to find the best-performing set of parameters. \emph{Performance} is given in terms of precision-recall graphs. 

\paragraph{(k-)Nearest Neighbor (JK).} The cross-validated precision-recall results for the k-NN classifier can be seen in \cref{fig:knn1}. Most generally, the classifier achieves very good performance on the data set.
As is usually expected, an initial increase in the $k$ numbers considered in the majority vote, increases the performance of classification, as this helps stabilize the prediction against outliers. However, when too many neighbors are considered, the predictive power of the classifier should decrease, since neighbors that are too far away, and are therefore likely to be of a different class, contribute to the majority vote. One can weigh the votes of each neighbor by the inverse of the distance from the instance to classify to somewhat remedy this problem.

The best performance is achieved at $k=9$ with 
\begin{align*}
	\text{accuracy} &= 0.97 \pm 0.02 \\
	\text{precision} &= 0.99 \pm 0.03 \\
	\text{recall} &= 0.93 \pm 0.05 \, .
\end{align*}
However the high noise from the cross-validation makes it entirely possible that other $k$ might achieve better performance on a test set. However, their performance will most likely be very similar to the one achieved with $k=9$, i.\ e.\ it does not really matter.

In this data set, we notice that the precision value of the classifier increases by 4 percentage points as we go to a 2-NN classifier. However, afterwards the data set is almost indifferent to an increase in $k$.  The recall value does slightly increase, however not significantly w.\ r.\ t.\ its standard deviation. This stability hints at the fact that meaningful features exist, clustering the data in a stable manner and therefore an increase in $k$ does not decrease performance.\footnote{As $k$ is increased, eventually instances of the opposite class are considered in the majority vote. However, if the data is approximately spherically clustered, an increasing number of correct instances is also considered. Therefore meaningful features yield some stability against large values of $k$.}
Only as we increase $k$ to large values of $k>50$ can a decrease in precision be observed, and finally the classifier breaks down at extreme and absolutely unreasonable $k$, i.\ e.\ the recall value increases because a lot of the positive instances are being falsely misclassified as negatives. The value drops below the orange guessing line, because the data set is imbalanced (more negative than positive instances) and at high $k$ this leads to a lot of false negatives, because the algorithm simply predicts the majority class. If the classification is redone with an artificially balanced data set the classification power stays above the worst case scenario of guessing even at high $k$.
However, at reasonable $k$ the imbalance in the data set does not affect the performance of the k-NN classifier, which makes sense, because imbalance is a global quantity whereas the k-NN classifier only looks at local balls in feature space.
\begin{figure}
	\centering
	 \begin{subfigure}[b]{0.45\textwidth}
	 	\includegraphics[width=\textwidth]{knn}
	 	\caption{}
	 	\label{fig:knn1}
    \end{subfigure}
	 \begin{subfigure}[b]{0.45\textwidth}
	 	\includegraphics[width=\textwidth]{knn_pca}
	 	\caption{}
	 	\label{fig:knn2}
    \end{subfigure}
	\caption{Precision-recall plot of the k-NN classifier for different values of k in the (a) original feature space and (b) the first four dimensions of the PCA. The grey line represents an ideal classifier, the orange line gives the performance for pure guessing, i.\ e.\ the worst case scenario.}
	\label{fig:knn}
\end{figure}

The k-NN algorithm has also been applied to the first four components of the PCA, which explain about \SI{80}{\percent} of the variance of the original data. A few differences can be achieved. Firstly, the performance of the PCA k-NN stays slightly below the performance of the raw k-NN. This is relieving, as our data set does not contain a lot of instances ($\approx 600$) in a space of 30 dimensions. One could have thought, that the performance of the classifier in the original feature breaks down due to the \emph{curse of high dimensions}. Compare to \cite{barber2012brm, bishop2006prm} for a textbook discussion of the topic.
Intuitively as the number of dimensions rises, the volume of a hypercube rises exponentially and therefore an exponential number of points are needed to cover the space densely. This means that high-dimensional data can very easily get sparse. It also affects the Euclidean distance metric: at high dimensions, the distance from one point to all it's neighbors becomes very similar. Even though the space is sparsely populated, most of the points are concentrated at a similar distance. 
This is why the raw k-NN precision recall curve seems to bunch together all $k\in {2,\dots, 20}$ while the PCA k-NN precision-recall curve is more spread out. In high dimensions, as we increase the number of neighbors taken into consideration, we do not actually venture further away from the point. Therefore, all of the points are bunched together, except for extreme values of $k$. The expected effect of increasing k (smooth curve with optimum at some $k$, and bad performance at extremal values) can only be observed for low dimensional data.
There is some discussion in the literature to what extend the performance of k-NN and related algorithms is necessarily affected by dimensionality. However, this is beyond the scope of this report and the reader is referred to \cite{marimont1979nearest, Chavez01searchingin, paper3}.

\paragraph{Support Vector Machines (JK).} The SVM was successfully applied to the data set. The results of the linear SVM can be taken from \cref{fig:svm1}. Compared to the nearest neighbor algorithm, the linear SVM shows an interesting behaviour as the trade-off parameter $C$ is increased. As $C$ is increased from small values, both precision and recall increase somewhat linearly until an optimal value is reached, from which the classifier then decays in a similar way as $C$ is increased further.

The optimal value of
\begin{align*}
	\text{accuracy} &= 0.98 \pm 0.01 \\
	\text{precision} &= 0.99 \pm 0.01 \\
	\text{recall} &= 0.96 \pm 0.03 \
\end{align*}
is reached at $C = 0.04$. As discussed in \cref{sec:theo}, low $C$ values express a preference of maximizing the margin and allowing misclassification in the training set. However, this shall not be overinterpreted as the results for higher $C$ values yield similar performance when accounting for uncertainty in the plots.

While at first glance the performance of this classification are better than the classification obtained by simpler classification algorithms, such as the nearest neighbor algorithm, it has to be noted, that the noise in the results does not allow such conclusions. It seems that in our comparison between the algorithms we are fundamentally limited by the noise in our data set. While an increase the samples size might help here, it could be that the noise is simply inherent to the data set.

\begin{figure}
	\centering
	 \begin{subfigure}[b]{0.45\textwidth}
	 	\includegraphics[width=\textwidth]{svm}
	 	\caption{}
	 	\label{fig:svm1}
    \end{subfigure}
	 \begin{subfigure}[b]{0.45\textwidth}
	 	\includegraphics[width=\textwidth]{svm_grid}
	 	\caption{}
	 	\label{fig:svm2}
    \end{subfigure}
	\caption{Precision-recall plot of the SVM classifier for different values of (a) the trade-off parameter $C$ for a linear SVM and (b) a grid of values $C$ and $\gamma$ for a kernelized SVM with a Gaussian kernel.}
	\label{fig:svm}
\end{figure}

Again, the artificially balanced data set did not result in better classifier performance. Although SVMs can suffer from undesired results when the data is imbalanced, this does not occur here in any way noticeable.
A kernel SVM with a Gaussian kernel was also applied and a grid search performed over possible combinations of $C$ and the kernel parameter $\gamma$.
The resulting cross-validated precision recall curve of this grid search can be seen in \cref{fig:svm2}.
There is no consistently apparent behavior of the precision-recall curve as either $C$ or $\gamma$ are varied. This is likely due to the fact that, limited by the computational complexity, the grid search only searches through orders of magnitudes for the desired parameters. A second round of grid search could have been performed after the first one, if the uncertainty in the data set were lower.

The best result, acknowledging that this is just a \emph{trend} and that the noise in the cross-validation is too high, was achieved at $(C=10^3, \gamma=10^{-4})$ with
\begin{align*}
	\text{accuracy} &= 0.98 \pm 0.02 \\
	\text{precision} &= 0.96 \pm 0.03 \\
	\text{recall} &= 0.98 \pm 0.02 \, .
\end{align*}

Again, the errors do not permit any detailed comparison between other algorithms, though it seems that the limited complexity of the data set is already well accounted for by a simple linear SVM. Note that for some extreme values not shown in the plot, e.\ g.\ low $C$ or high $\gamma$, the classification performance broke down completely.

\paragraph{Multilayer Perceptron (JK).} Neural networks have many hyperparameters such as learning rate (how much do we adjust the parameters after each backpropagation), batch size (how many instances do we feed into the network before we backpropagate), epochs (how many times do we propagate all of the data through the network), as well as the general architecture of the network. Additionally, training large neural network is computationally expensive, which is why grid searches over the hyperparameterspace as well as cross-validation are rarely seen in current publications.\footnote{Plus, in this report, I see the MLP more as a pet project, a nice little extra, whose performance need not to be pushed to the maximum.}

A simple MLP with \SI{30} input nodes, \SI{300} nodes in a single hidden layer, and \SI{2} output nodes was constructed. Training for \SI{200} epochs with \SI{50} instances per batch and a learning rate of \SI{0.05} yielded a test set performance of 
\begin{align*}
	\text{accuracy} &= 0.98 \\
	\text{precision} &= 0.97  \\
	\text{recall} &= 0.98  \, .
\end{align*}
However, due to the reasons mentioned above no cross validation was performed, and the values are therefore without error. Due to the stochastic nature of the batch training, the values vary slightly between different sessions. However, even though no hyperparameter optimization was performed, the test set classification performance of a simple MLP is fairly high and comparable the other presented algorithms.

\section{Discussion}
\label{sec:discu}

%TODO unfortunately original images are not available. otherwise could have trained CNN to do it all automatically just like thorsten did in his paper
%TODO essentially not comparable with the high noise present
%TODO read up, there was some way to make cross validation less noisy, that we could have used


\section{Summary}

%TODO most / all? our algorithms reached performance atleast par with what original authors published

\section{Appendix}


\bibliographystyle{ieeetr}
\bibliography{bibliography}

\end{document}